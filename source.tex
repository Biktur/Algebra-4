\documentclass{article}
\usepackage[T2A]{fontenc}
\usepackage[english, russian]{babel}
\usepackage{mathtools, amssymb, amsfonts, amsthm}
\usepackage{tikz-cd}

\newtheorem*{theorem}{Теорема}
\newtheorem*{definition}{Определение}
\newtheorem*{proposition}{Предложение}

\DeclareMathOperator{\chr}{char}
\DeclareMathOperator{\End}{End}
\DeclareMathOperator{\im}{Im}

\title{}
\date{}

\begin{document}
\maketitle

Ленг. Алгебра
Конспект Степанова
Представления и характеры конечных групп

\begin{theorem}[Бёрнсайд]
Группа порядка $p^nq^m$ разрешима.
\end{theorem}

\begin{theorem}[Машке]
$k$ --- поле, $G$ --- группа, $\chr k \nmid |G|$, $V$ --- $k[G]$-модуль, $u$ -- подмодуль. Тогда найдётся подмодуль $W$, т. ч.  $V = U \oplus W$.
\end{theorem}

\begin{definition}
Неприводимый $G$-модуль --- простой $k[G]$-модуль.
\end{definition}

\begin{definition}
Неразложимый --- если $M = M_1 \oplus M_2$, то  $M_1$ или $M_2$ тривиален.
\end{definition}

Везде далее $V$ --- конечномерное векторное пространство над $k$, а $\chr k \nmid |G|$.

Из конечномерноссти, разлоожение в прямую сумму неприводимых, очевидно существует, хотим доказать единственность.

\begin{theorem}[Шур]
Пусть $M, N$ --- простые $R$-модули. Тогда
\begin{enumerate}
\item Любой гомоморфизм $M \rightarrow^\phi N$ либо тривиален, либо является изоморфизмом.
\item $\End_k(M)$ --- тело*.
\end{enumerate}
\end{theorem}
\begin{proof}
\begin{enumerate}
\item Пусть $\phi$ не изоморфизм. Тогда либо $\ker \phi \neq 0$, либо $\im \phi \neq N$. В первом случае $\ker \phi = M$, во втором --- $\im \phi = 0$, т. е. $\phi = 0$.
\item Ну действительно, любой ненулевой элемент --- изоморфизм, а значит обратим.
\end{enumerate}
\end{proof}
Вернёмся к доказательству единственности.
	Пусть $V \simeq M_1^{r_1} \oplus M_2^{r_2} \oplus \ldots \oplus M_k^{r_k}$ и $V \simeq N_1^{s_1} \oplus N_2^{s_2} \oplus \ldots \oplus N_l^{s_l}$, где $M_i \not \simeq M_j$ и $N_i \not \simeq N_j$ при разлчиных $i, j$.

	Пусть $M_i \not \simeq N_j$. Тогда гомоморфизм $M_i^{r_i} \rightarrow^\phi N_j^{r_j}$ определённый понятным образом, тривиален. 

	Пусть теперь $M_i \simeq N_j$. Из вышесказанного $\phi$, очевидно, инъективен, но так же определено и отображение в обратну сторону, но тогда размернсти $M_i^{r_i}$ и $N_j^{s_j}$ совпадают, в частности, из соображений размерности $r_i = s_j$.

\begin{definition}
Кольцо $R$ называется $k$-алгеброй. Если $R$ - векторное пространоство над $k$, причём $\lambda(xy) = (\lambda x)y = x(\lambda y)$.
\end{definition}

Из теоремы выше,

	$k[G] \simeq L_1^{r_1} \oplus L_2^{r_2} \oplus \ldots \oplus L_n^{r_n}$, $L_i$ --- простые и $L_i \not \simeq L_j$, при $i \neq j$.

\begin{proposition} %Утверждение
Пусть $L$ --- простой идеал в $R$. $M$ --- простой модуль. Тогда если $L \not \simeq M$, то $LM = 0$.
\end{proposition}
	\begin{proof}
	для любого $m$ определено отображение $l \mapsto m$. По лемме Шура, все такие отображения тривиальны, т. е. $LM = 0$.
	\end{proof}

	Пусть $1 = e_1 + e_2 + \ldots + e_m$, где $e_i \in L_i^{r_i}$. Отсюда сразу же 
$e_ie_j = 0$ при $i \neq j$, и $e_i^2 = e_i$. Кроме того, очевидно, $e_i \in Z(k[G])$.
Если $a_1e_1 + a_2e_2 + \ldots + a_ne_n = 0$, домножая на $e_i$ получаем $a_i = 0$. Значит $m \leq \dim_kZ(k[G])$.
\begin{theorem}
Пусть $M$ --- простой $G$-модуль. Тогда для какого-то $i$, $L_i \simeq M$.
\end{theorem}
	\begin{proof}
	для какого-то $i$, $L_iM \neq 0$. Отсюда $L_i \simeq M$.
	\end{proof}

Для $g \in G$ рассмотрим $K_g = \{hgh^{-1} \mid h \in G\}$.
Обозначим $\hat{K}_g = \sum_{\sigma \in K_g}\sigma$.
\begin{theorem}
$\{\hat{K}_g\}_{g \in I}$, где $I$ --- множество представителей классов сопряжённости, сосатвляют базис $Z(k[G])$.
\end{theorem}
$\hat{K}_g$, очевидно, лежит в центре при любом $g \in G$. С другой стороны, если
	$x = \sum_\sigma a_\sigma \sigma \in Z(k[G])$,
	$x = \sum_\sigma a_\sigma \tau \sigma \tau^{-1}$.

\end{document}

